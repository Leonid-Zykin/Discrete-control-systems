% Основная часть отчёта по ЛР№1: Дискретные системы управления

\chapter{Исследование влияния дискретного элемента на непрерывную систему}
\section{Постановка задачи}
Вариант: \textbf{8}. Для схемы на рис.~\ref{fig:task1_scheme} заданы параметры: период дискретизации \(T=0{,}2\,\text{с}\), усиление непрерывной части \(K_{CO}=3{,}4\). Требуется:
\begin{enumerate}
  \item[\textbf{(a)}] Реализовать схему в среде моделирования (Python/NumPy/Matplotlib). Для дискретного звена использовать экстраполятор нулевого порядка (ZOH).
  \item[\textbf{(b)}] Подбором коэффициента обратной связи \(K_{FB}\) найти границы устойчивости (нейтральная и колебательная) замкнутой системы. Построить переходные характеристики выхода.
  \item[\textbf{(c)}] Сделать вывод о влиянии ZOH на устойчивость замкнутой системы.
  \item[\textbf{(d)}] Исследовать влияние \(K_{FB}\) на колебательность процесса: найти значения, соответствующие максимальной колебательности и отсутствию колебаний; построить переходные процессы.
  \item[\textbf{(e)}] Найти значение \(K_{FB}\), обеспечивающее оптимальный по быстродействию процесс; представить переходные характеристики.
\end{enumerate}

\begin{figure}[H]
  \centering
  \includegraphics{task1/scheme_placeholder}
  \caption{Структурная схема моделирования задания 1 (иллюстрация из методички).}
  \label{fig:task1_scheme}
\end{figure}

\section{Математическая модель}
Непрерывная часть имеет передаточную функцию вида
\[
  W_c(s) = \frac{K_{CO}}{s}, \quad K_{CO}=3{,}4.
\]
При ZOH-дискретизации и замыкании по \(K_{FB}\) дискретная динамика для состояния интегратора описывается
\[
  x_{k+1} = \bigl(1 - T K_{CO} K_{FB}\bigr)\,x_k + T K_{CO},\qquad y_k = x_k,
\]
где собственное число замкнутой системы \(a = 1 - T K_{CO} K_{FB}\).

\section{Ход моделирования}
Реализация выполнена в скрипте \texttt{python/task1.py}. Скрипт формирует переходные процессы для различных значений \(K_{FB}\) и сохраняет рисунки в папку \texttt{images/task1/}.

\subsection*{(b) Границы устойчивости}
Границы по условию \(|a|=1\) дают \(K_{FB}=0\) (нейтральная) и \(K_{FB}=\tfrac{2}{T K_{CO}}=2/(0{,}2\cdot3{,}4)\approx2{,}941\) (колебательная, \(a=-1\)).
\begin{figure}[H]
  \centering
  \includegraphics{task1/step_boundary_neutral}
  \caption{Переходная характеристика при нейтральной границе устойчивости (\(K_{FB}=0\)).}
  \label{fig:task1_neutral}
\end{figure}
\begin{figure}[H]
  \centering
  \includegraphics{task1/step_boundary_osc}
  \caption{Переходная характеристика при колебательной границе устойчивости (\(a=-1\), \(K_{FB}\approx2{,}941\)).}
  \label{fig:task1_osc}
\end{figure}

\subsection*{(c) Влияние ZOH}
ZOH фиксирует управляющее воздействие на интервале дискретизации, что эквивалентно появлению дискретного собственного числа \(a=1-TK_{CO}K_{FB}\). В результате устойчивость определяется положением \(a\) внутри единичного круга; чем ближе \(a\) к границе \(-1\), тем больше колебательность.

\subsection*{(d) Влияние коэффициента обратной связи}
\begin{figure}[H]
  \centering
  \includegraphics{task1/step_no_osc}
  \caption{Переходная характеристика без колебаний (\(0<a<1\)).}
  \label{fig:task1_no_osc}
\end{figure}
\begin{figure}[H]
  \centering
  \includegraphics{task1/step_max_osc}
  \caption{Переходная характеристика при максимальной колебательности (\(a\approx-0{,}9\)).}
  \label{fig:task1_max_osc}
\end{figure}
Тенденции: при уменьшении \(a\) в диапазоне \((0,1)\) процесс становится быстрее и апериодичнее; при отрицательных \(a\) появляется колебательность, её амплитуда растёт по мере приближения \(a\) к \(-1\).

\subsection*{(e) Оптимальный по быстродействию процесс}
\begin{figure}[H]
  \centering
  \includegraphics{task1/step_fast}
  \caption{Оптимальный по быстродействию переходный процесс (пример \(a=0{,}1\)).}
  \label{fig:task1_fast}
\end{figure}
Выбор малого положительного \(a\) обеспечивает быстрое затухание, сохраняя апериодический характер ответа и умеренные усилия управления.

\section{Выводы по заданию 1}
ZOH делает замкнутую систему дискретной с собственным числом \(a=1-T\,K_{CO}\,K_{FB}\). Границы устойчивости соответствуют \(|a|=1\): \(K_{FB}=0\) и \(K_{FB}=2/(T K_{CO})\). При \(0<a<1\) процесс апериодический; при \(-1<a<0\) — колебательный, степень колебательности растёт при приближении к \(-1\). Выбор меньшего \(a\) ускоряет процесс, но повышает требования к управляющему воздействию; слишком малые \(a\) могут приводить к насыщению исполнительных органов.

\chapter{Исследование устойчивости дискретных систем}
\section{Постановка задачи}
Сформировать дискретную модель системы \(\ddot{y} = u\) при ZOH-дискретизации: \(A_d=\begin{bmatrix}1 & T\\0 & 1\end{bmatrix},\ B_d=\begin{bmatrix}T^2/2\\T\end{bmatrix}\). Задать управление \(u(k)=-Kx(k)=-[k_1\;k_2]x(k)\). По пяти наборам желаемых корней из таблицы варианта 8 синтезировать \(K\), рассчитать матрицу \(F=A_d-B_dK\) и выполнить моделирование при исходных условиях \(y(0)=1,\ \dot{y}(0)=0\).

\section{Результаты расчётов и моделирования}
Расчёты выполнены в скрипте \texttt{python/task2.py} (алгоритм Аккермана). Полученные переходные процессы приведены на рис.~\ref{fig:task2_set1}–\ref{fig:task2_set5}. Итоговые коэффициенты \(K=[k_1\;k_2]\):

\begin{table}[H]
  \centering
  \begin{tabular}{cccc}
    \toprule
    Набор & Полюса & $k_1$ & $k_2$ \\
    \midrule
    1 & $\{0.8,\ 0.2\}$ & $-21.0$ & $-2.9$ \\
    2 & $\{1.0,\ -0.3\}$ & $-25.0$ & $-1.0$ \\
    3 & $\{0.6,\ -0.3\}$ & $-12.0$ & $-0.3$ \\
    4 & $\{0.7j,\ -0.7j\}$ & $\phantom{-}12.25$ & $-1.225$ \\
    5 & $\{-0.3\!+\!0.8j,\ -0.3\!-\!0.8j\}$ & $\phantom{-}33.25$ & $-0.325$ \\
    \bottomrule
  \end{tabular}
  \caption{Коэффициенты регулятора состояния по пяти наборам желаемых корней.}
\end{table}

Качественный анализ:
\begin{itemize}
  \item \textbf{Наборы 1 и 3 (действительные полюса в пределах круга)}: апериодическое затухание; чем меньше модули полюсов, тем быстрее процесс и короче время установления.
  \item \textbf{Набор 2 (полюс при 1)}: крайняя медлительность из‑за близости к границе устойчивости; заметное затягивание перехода.
  \item \textbf{Наборы 4 и 5 (комплексные пары)}: колебательный характер; увеличение радиуса или уменьшение затухания приводит к большему перерегулированию и длительным колебаниям.
\end{itemize}

\begin{figure}[H]
  \centering
  \includegraphics{task2/set1_step}
  \caption{Набор 1: переходный процесс.}
  \label{fig:task2_set1}
\end{figure}
\begin{figure}[H]
  \centering
  \includegraphics{task2/set2_step}
  \caption{Набор 2: переходный процесс.}
  \label{fig:task2_set2}
\end{figure}
\begin{figure}[H]
  \centering
  \includegraphics{task2/set3_step}
  \caption{Набор 3: переходный процесс.}
  \label{fig:task2_set3}
\end{figure}
\begin{figure}[H]
  \centering
  \includegraphics{task2/set4_step}
  \caption{Набор 4: переходный процесс.}
  \label{fig:task2_set4}
\end{figure}
\begin{figure}[H]
  \centering
  \includegraphics{task2/set5_step}
  \caption{Набор 5: переходный процесс.}
  \label{fig:task2_set5}
\end{figure}

\section*{Выводы по заданию 2}
Размещение корней позволяет напрямую задать желаемые динамические показатели. Действительные корни ближе к нулю дают быстрое апериодическое поведение, комплексные корни — колебательный процесс; приближение полюсов к единичной окружности замедляет систему и повышает чувствительность к возмущениям.

\chapter{Построение дискретных командных генераторов}
\section{Генератор гармонического сигнала}
Реализован генератор \(g(k) = A\sin(kT\omega)\) через вращающуюся систему второго порядка. Такой генератор удобен как эталонное воздействие: частота настраивается параметром \(\omega\), дискретизация — периодом \(T\).
\begin{figure}[H]
  \centering
  \includegraphics{task3/gen_harmonic}
  \caption{Генератор гармонического сигнала для параметров варианта 8.}
\end{figure}

\section{Математическая модель возмущения}
Вариант 8: \(4\sin(2kT) + 1.5\cos(2.5kT)\). Сумма двух автономных осцилляторов позволяет использовать модель как вход «возмущение» в схемах «вход–состояние–выход» для оценки устойчивости/робастности. Период дискретизации для модели возмущения задан \(T=0{,}25\,\text{с}\) согласно подзаданию (d).
\begin{figure}[H]
  \centering
  \includegraphics{task3/disturbance}
  \caption{Выход дискретной модели возмущения.}
\end{figure}

\section*{Выводы по заданию 3}
Построенные генераторы обеспечивают воспроизводимую подачу тестовых сигналов и возмущений для дискретных систем с заданным периодом дискретизации, что позволяет сравнивать поведение различных регуляторов при одинаковых условиях.
