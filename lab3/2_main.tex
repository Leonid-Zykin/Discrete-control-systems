% Основная часть отчёта по лабораторной работе №3

\chapter{Постановка задачи}
Цель работы: исследовать цифровую САУ температуры, синтезировать и настроить дискретный регулятор согласно порядку выполнения из методички (стр. 49–56). Вариант: \textbf{8} (табл.~7: $T_1=0{,}9$, $T_2=1{,}05$).

\section{Модель объекта и выбор периода дискретизации}
Непрерывная часть объекта представляется последовательно соединёнными звеньями вида $\frac{1}{T_1 s + 1}$ и $\frac{1}{T_2 s + 1}$.
Период дискретизации выбираем по заданию: сначала $T = T_1/2$, затем $T=T_1/4$. В качестве датчика используем ZOH.

\section{Синтез дискретного регулятора}
Для структуры согласно рис.~17–18 методички выполняется расчёт параметров регулятора и подбор коэффициента передачи $q_0$, обеспечивающего слабоколебательный процесс.

\subsection{Расчёт параметров регулятора}
Непрерывная передаточная функция объекта:
\[
G(s) = \frac{1}{(T_1 s + 1)(T_2 s + 1)} = \frac{1}{T_1 T_2 s^2 + (T_1 + T_2) s + 1}.
\]
Для варианта 8: $T_1 = 0{,}9$~с, $T_2 = 1{,}05$~с.

Дискретизация через ZOH (zero-order hold) при периоде дискретизации $T$:
\[
G(z) = \frac{B(z)}{A(z)} = \mathcal{Z}\left\{ \frac{1 - e^{-Ts}}{s} \cdot G(s) \right\},
\]
где $\mathcal{Z}$ обозначает $z$-преобразование с учётом фиксатора нулевого порядка.

Структура дискретного регулятора согласно схеме (рис.~18):
\[
W(z) = q_0 \cdot \frac{\text{num}(z)}{z^2 - z} = q_0 \cdot \frac{\text{num}(z)}{z(z-1)},
\]
где числитель $\text{num}(z)$ компенсирует полюса объекта $A(z)$, обеспечивая требуемое качество переходных процессов.

После дискретизации объекта для $T = T_1/2 = 0{,}45$~с получена передаточная функция вида:
\[
G(z) = \frac{b_1 z^{-1} + b_2 z^{-2}}{1 + a_1 z^{-1} + a_2 z^{-2}}.
\]
Параметры регулятора рассчитываются так, чтобы числитель $\text{num}(z)$ компенсировал знаменатель $A(z)$ объекта, что обеспечивает требуемую динамику замкнутой системы.

\subsection{Подбор коэффициента $q_0$}
Методика подбора $q_0$ (в соответствии с п.~2–3 порядка работ):
\begin{enumerate}
    \item фиксируем период дискретизации $T$ (сначала $T=T_1/2$, затем $T=T_1/4$) и дискретизуем объект через ZOH;
    \item рассчитываем параметры регулятора для компенсации полюсов объекта;
    \item проводим серию моделирований по сетке $q_0\in[q_{\min}, q_{\max}]$;
    \item выбираем $q_0$, обеспечивающий перерегулирование в пределах 5–15\% и минимальное время установления (если таких нет, берём минимум по интегральному критерию ISE при устойчивости).
\end{enumerate}
По результатам подбора: при $T=T_1/2$ получено $q_0\approx4.00$, при $T=T_1/4$ — $q_0\approx4.92$.

\section{Эксперименты}
Для каждого из периодов дискретизации проведены три эксперимента:
\begin{enumerate}
    \item реакция на ступенчатое задающее воздействие;
    \item реакция на ступенчатое возмущающее воздействие;
    \item реакция на возмущение, изменяющееся по случайному закону.
\end{enumerate}

\begin{figure}[H]
    \centering
    \includegraphics{task1/step_set_T12.png}
    \caption{Ступенчатое задание, $T=T_1/2$}
\end{figure}

\begin{figure}[H]
    \centering
    \includegraphics{task1/step_dist_T12.png}
    \caption{Ступенчатое возмущение, $T=T_1/2$}
\end{figure}

\begin{figure}[H]
    \centering
    \includegraphics{task1/noise_T12.png}
    \caption{Случайное возмущение, $T=T_1/2$}
\end{figure}

Для $T=T_1/4$ получены аналогичные результаты (см. ниже):

\begin{figure}[H]
    \centering
    \includegraphics{task1/step_set_T14.png}
    \caption{Ступенчатое задание, $T=T_1/4$}
\end{figure}

\begin{figure}[H]
    \centering
    \includegraphics{task1/step_dist_T14.png}
    \caption{Ступенчатое возмущение, $T=T_1/4$}
\end{figure}

\begin{figure}[H]
    \centering
    \includegraphics{task1/noise_T14.png}
    \caption{Случайное возмущение, $T=T_1/4$}
\end{figure}

\section{Влияние периода дискретизации и неточности $T_2$}
Сравниваются качества процесса управления при $T=T_1/2$ и $T=T_1/4$ (Рисунок ниже): уменьшение $T$ снижает дискретизационные искажения и ускоряет процесс, цена — более частое обновление управления. Также исследуется влияние неточности компенсации полюсов: $T_2$ изменяется на $\pm20\%$, реакции фиксируются при неизменном $q_0$ для режима $T=T_1/2$. При занижении/завышении $T_2$ наблюдаются соответственно более быстрые/замедленные переходные и изменение перерегулирования — это демонстрирует чувствительность САУ к ошибок идентификации.

\begin{figure}[H]
    \centering
    \includegraphics{task2/compare_T.png}
    \caption{Сравнение $T=T_1/2$ и $T=T_1/4$ (реакция на ступенчатое возмущение)}
\end{figure}

\begin{figure}[H]
    \centering
    \includegraphics{task3/compare_T2_perturb.png}
    \caption{Влияние ошибки $T_2$ на реакцию на возмущение}
\end{figure}

\noindent Для выполнения п.~5 задания параметры регулятора были \textbf{пересчитаны и установлены} для двух случаев $T_2\,\pm20\%$ при фиксированном $T=T_1/2=0{,}45$~с. Согласно методике
\[
    W_c(z) = q_0 \cdot \frac{(z-d_1)(z-d_2)}{z(z-1)},\quad d_1=e^{-T/T_1},\; d_2=e^{-T/T_2}.
\]
Численно: $d_1=e^{-0.45/0.9}=e^{-0.5}\approx\mathbf{0{,}6065}$. Тогда
\begin{itemize}
    \item $T_2=1{,}05$ (номинал): $d_2\approx\mathbf{0{,}6514}$, $\text{num}(z)=z^2-\mathbf{1{,}2580}z+\mathbf{0{,}3951}$;
    \item $T_2=0{,}84$ ($-20\%$): $d_2\approx\mathbf{0{,}5856}$, $\text{num}(z)=z^2-\mathbf{1{,}1921}z+\mathbf{0{,}3556}$;
    \item $T_2=1{,}26$ ($+20\%$): $d_2\approx\mathbf{0{,}6990}$, $\text{num}(z)=z^2-\mathbf{1{,}3055}z+\mathbf{0{,}4232}$.
\end{itemize}
Знаменатель для всех случаев одинаков: $z(z-1)$. Коэффициент $q_0$ принят из п.~3. На рисунке выше приведены зафиксированные процессы на выходе регулятора и системы при ступенчатом возмущении для трёх значений $T_2$.

\section{Выводы}
Уменьшение периода дискретизации до $T=T_1/4$ даёт более короткие переходные и меньшую интегральную ошибку при сопоставимом перерегулировании по сравнению с режимом $T=T_1/2$. Подбор коэффициента $q_0$ позволяет получить слабоколебательные переходные процессы. Изменение постоянной времени $T_2$ на $\pm20\%$ заметно отражается на быстродействии и величине перерегулирования (при увеличении $T_2$ процесс замедляется, при уменьшении — ускоряется), поэтому при настройке регулятора важна корректная идентификация параметров объекта.


