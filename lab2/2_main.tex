% Основная часть отчёта по лабораторной работе №2

\chapter{Постановка задачи}
На страницах \textit{24--48} методички \textit{Дискретные системы управления} приведены задания по синтезу классических регуляторов для дискретных систем. Вариант: \textbf{8}. Требуется выполнить три блока работ: 
\begin{enumerate}
    \item синтез стабилизирующего регулятора для заданного типа ОУ и проверка свойств замкнутой системы; 
    \item синтез следящего регулятора методом внутренней модели для выбранного сигнала задания; 
    \item построение наблюдателя состояния и моделирование системы при неполной измеряемости.
\end{enumerate}

Параметры варианта (табл. 5, 6 методички):
\begin{itemize}
    \item тип ОУ: 4; $k_1=3.20$, $a_0^1=0$, $T_1=1$, $a_0^1=0$; $k_2=1$, $a_0^2=0$, $T_2=2$; период дискретизации $T=0.75$;
    \item сигнал задания: гармонический, $A_g=2.06$, $\omega_g=7$ (табл. 6, вар. 8).
\end{itemize}

\section{Задание 1. Стабилизирующий регулятор}
\subsection{Непрерывная модель и дискретизация}
Определим структуру объекта по рисунку 13 (тип 4) и параметрам варианта, затем выполним дискретизацию с периодом $T=0.75$. 

Непрерывная модель в пространстве состояний:
\[\label{eq:cont_model}
\dot x = A_c x + B_c u,\quad y = C x,\quad 
A_c = \begin{bmatrix}0 & 0\\ k_1 & 0\end{bmatrix},\; B_c = \begin{bmatrix}1\\0\end{bmatrix},\; C=\begin{bmatrix}0 & 1\end{bmatrix}.
\]
где $k_1=3.20$ для варианта 8. Матрица $A_c$ имеет собственные значения $\lambda_1=\lambda_2=0$, что указывает на критическую устойчивость (два интегратора).

Дискретизация по нулевому порядку (ZOH):
\[\label{eq:disc_model}
 x_{k+1} = A_d x_k + B_d u_k,\quad A_d = e^{A_c T},\; B_d = \int\limits_0^{T} e^{A_c \tau}\, d\tau\, B_c.
\]
Для матрицы $A_c$ с нулевыми собственными значениями:
\[
e^{A_c T} = I + A_c T + \frac{(A_c T)^2}{2!} + \cdots = \begin{bmatrix}1 & 0\\ k_1 T & 1\end{bmatrix}
\]
\[
B_d = \int\limits_0^{T} \begin{bmatrix}1 & 0\\ k_1 \tau & 1\end{bmatrix} \begin{bmatrix}1\\0\end{bmatrix} d\tau = \begin{bmatrix}T\\ \frac{k_1 T^2}{2}\end{bmatrix}
\]
При $T=0.75$ и $k_1=3.20$:
\[
A_d = \begin{bmatrix}1 & 0\\ 2.4 & 1\end{bmatrix},\quad B_d = \begin{bmatrix}0.75\\ 0.9\end{bmatrix}
\]

Структурные свойства проверялись по стандартным критериям: 
\[\text{rank}\, \mathcal C = \text{rank}\,\bigl[ B_d\; A_d B_d\; \dots \; A_d^{n-1} B_d \bigr] = n,\qquad 
\text{rank}\, \mathcal O = \text{rank}\,\begin{bmatrix} C\\ C A_d \\ \vdots \\ C A_d^{n-1}\end{bmatrix}= n.
\]
Для данного случая:
\[
\mathcal C = \begin{bmatrix}0.75 & 0.75\\ 0.9 & 2.7\end{bmatrix},\quad \det\mathcal C = 0.75 \cdot 2.7 - 0.9 \cdot 0.75 = 1.35 \neq 0
\]
\[
\mathcal O = \begin{bmatrix}0 & 1\\ 2.4 & 1\end{bmatrix},\quad \det\mathcal O = 0 \cdot 1 - 1 \cdot 2.4 = -2.4 \neq 0
\]
Система полностью управляема и наблюдаема.

На Рисунке~\ref{fig:task1_open} показан отклик разомкнутой системы на единичное ступенчатое воздействие. Из-за наличия двух интеграторов наблюдается неограниченный рост выхода (неустойчивость в разомкнутом виде), что соответствует теоретическому ожиданию для типа 4.

\begin{figure}[H]
    \centering
    \includegraphics{task1/plant_step_open.png}
    \caption{Переходная характеристика ОУ (разомкнутая система)}
    \label{fig:task1_open}
\end{figure}

\subsection{Синтез регулятора}
Требуем оптимальные по быстродействию корни дискретной системы: $z_i^*=0$. Выполним модальный синтез обратных связей для дискретной модели состояния (формула Акерманна для SISO).

Для одноканальной системы $x_{k+1}=A_d x_k + B_d u_k$, $u_k=-Kx_k$ используем формулу Акерманна:
\[\label{eq:ack}
K = e_n^T \mathcal C^{-1}\, \phi(A_d),\quad \phi(\lambda)=\lambda^{n}+a_{n-1}\lambda^{n-1}+\cdots+a_0,
\]
где $\phi$~--- желательный характеристический многочлен ($\phi(\lambda)=\lambda^n$ для $z_i^*=0$), $e_n^T = [\,0\;\dots\;0\;1\,]$.

Для deadbeat-регулятора $\phi(\lambda)=\lambda^2$, поэтому:
\[
\phi(A_d) = A_d^2 = \begin{bmatrix}1 & 0\\ 2.4 & 1\end{bmatrix}^2 = \begin{bmatrix}1 & 0\\ 4.8 & 1\end{bmatrix}
\]
\[
K = \begin{bmatrix}0 & 1\end{bmatrix} \mathcal C^{-1} \phi(A_d) = \begin{bmatrix}0 & 1\end{bmatrix} \begin{bmatrix}0.75 & 0.75\\ 0.9 & 2.7\end{bmatrix}^{-1} \begin{bmatrix}1 & 0\\ 4.8 & 1\end{bmatrix}
\]
\[
\mathcal C^{-1} = \frac{1}{1.35}\begin{bmatrix}2.7 & -0.75\\ -0.9 & 0.75\end{bmatrix} = \begin{bmatrix}2 & -0.556\\ -0.667 & 0.556\end{bmatrix}
\]
\[
K = \begin{bmatrix}0 & 1\end{bmatrix} \begin{bmatrix}2 & -0.556\\ -0.667 & 0.556\end{bmatrix} \begin{bmatrix}1 & 0\\ 4.8 & 1\end{bmatrix} = \begin{bmatrix}2 & 0.556\end{bmatrix}
\]

Полученный в расчёте вектор усилений: $K=[2\;\;0.5556]$. 

\textbf{Уравнение стабилизирующего регулятора:}
\[
u_k = -K x_k = -2 x_{1,k} - 0.5556 x_{2,k}
\]
где $x_{1,k}$, $x_{2,k}$ — компоненты вектора состояния на $k$-м такте.

На Рисунке~\ref{fig:task1_closed} показан отклик замкнутой системы при нулевом входе и ненулевом начальном состоянии: наблюдается \textit{deadbeat}-схождение за конечное число тактов, что подтверждает размещение корней в нуле.

\begin{figure}[H]
    \centering
    \includegraphics{task1/closed_step.png}
    \caption{Переходная характеристика замкнутой системы с синтезированным регулятором}
    \label{fig:task1_closed}
\end{figure}

\section{Задание 2. Следящий регулятор (метод внутренней модели)}
Синтезируем генератор задания $g(k)=A_g\sin(\omega_g k T)$ с параметрами варианта и включим его во внутреннюю модель (резонатор второго порядка). Синтезируем регулятор на расширенной системе методом модального управления с размещением всех дискретных корней в нуле.

Дискретная внутренняя модель синусоидального сигнала реализуется как резонатор
\[
 w_{k+1} = A_{\text{osc}} w_k + B_{\text{osc}} e_k,\quad e_k = r_k - y_k,
\]
\[
 A_{\text{osc}} = \begin{bmatrix}0 & 1\\ -1 & 2\cos(\omega_g T)\end{bmatrix},\quad B_{\text{osc}}=\begin{bmatrix}0\\1\end{bmatrix}.
\]
При $\omega_g=7$ и $T=0.75$: $\cos(\omega_g T) = \cos(5.25) \approx -0.515$, поэтому:
\[
A_{\text{osc}} = \begin{bmatrix}0 & 1\\ -1 & -1.03\end{bmatrix}
\]

Расширенная система $z=[x^T\; w^T]^T$ имеет вид
\[
 z_{k+1}= \underbrace{\begin{bmatrix} A_d & 0\\ -B_{\text{osc}} C & A_{\text{osc}}\end{bmatrix}}_{A_t} z_k + \underbrace{\begin{bmatrix} B_d \\ 0 \end{bmatrix}}_{B_t} u_k + \begin{bmatrix}0\\ B_{\text{osc}}\end{bmatrix} r_k,
\]
где:
\[
A_t = \begin{bmatrix}
1 & 0 & 0 & 0\\
2.4 & 1 & 0 & 0\\
0 & -1 & 0 & 1\\
0 & 0 & -1 & -1.03
\end{bmatrix},\quad B_t = \begin{bmatrix}0.75\\ 0.9\\ 0\\ 0\end{bmatrix}
\]

Закон управления $u_k=-\overline{K} \overline{x}_k$, где матрица $\overline{K}=[K_\eta\;K_x]$ найдена по формуле Акерманна для расширенной системы. Для deadbeat-регулятора $\phi(\lambda)=\lambda^4$:
\[
\overline{K} = e_n^T \mathcal C^{-1} \phi(\overline{A})
\]
где $\overline{A}$ — матрица расширенной системы, $\mathcal C$ — матрица управляемости расширенной системы.

Полученные собственные значения замкнутой системы: $\lambda_1 = 1.82 \times 10^{-4}$, $\lambda_2 = 6.61 \times 10^{-10} + 1.82 \times 10^{-4}j$, $\lambda_3 = 6.61 \times 10^{-10} - 1.82 \times 10^{-4}j$, $\lambda_4 = -1.82 \times 10^{-4}$. Все полюса практически в нуле, что обеспечивает почти идеальный deadbeat-режим с теоретическим временем затухания 0.35 секунды.
Матрица управляемости расширенной системы:
\[
\mathcal C_t = \begin{bmatrix}B_t & A_t B_t & A_t^2 B_t & A_t^3 B_t\end{bmatrix}
\]
После вычислений получены усиления $K_x=[2.4462\;\;1.3217]$, $K_w=[0.3943\;\;0.3399]$.

\textbf{Уравнение следящего регулятора с внутренней моделью:}
\[
u_k = -\overline{K} \overline{x}_k = -K_\eta \eta_k - K_x x_k
\]
где $\overline{x}_k = [\eta_k; x_k]$ — расширенный вектор состояния, $\eta_k = [\eta_{1,k}, \eta_{2,k}]^T$ — состояние внутренней модели, $x_k = [x_{1,k}, x_{2,k}]^T$ — состояние объекта.

Полученные коэффициенты регулятора:
\begin{itemize}
    \item $K_\eta = [0.3943, 0.3399]$ — коэффициенты внутренней модели
    \item $K_x = [2.4462, 1.3217]$ — коэффициенты объекта
\end{itemize}

На Рисунках~\ref{fig:task2_ref}--\ref{fig:task2_resp} видно, что выход $y(k)$ следует за гармоническим заданием с исчезающей установившейся ошибкой. Deadbeat-синтез обеспечивает сходимость за 4 такта (3 секунды). Детальный анализ показывает:
\begin{itemize}
    \item Такт 0: $t=0.00$с, ошибка $=0.000000$ (начальное состояние)
    \item Такт 1: $t=0.75$с, ошибка $=-1.769405$ (первый пик)
    \item Такт 2: $t=1.50$с, ошибка $=-1.812173$ (максимальная ошибка)
    \item Такт 3: $t=2.25$с, ошибка $=-0.627856$ (затухание)
    \item Такт 4: $t=3.00$с, ошибка $=0.000000$ (точная сходимость)
\end{itemize}
Максимальная ошибка в переходном процессе составляет 1.81, установившаяся ошибка точно нулевая. Теоретическое время затухания составляет 0.35 секунды, что значительно лучше требуемых 3 секунд.

\begin{figure}[H]
    \centering
    \includegraphics{task2/reference.png}
    \caption{Задающее воздействие $g(k)$}
    \label{fig:task2_ref}
\end{figure}

\begin{figure}[H]
    \centering
    \includegraphics{task2/servo_response.png}
    \caption{Выход системы и ошибка слежения $e(k)=g(k)-y(k)$}
    \label{fig:task2_resp}
\end{figure}

\section{Задание 3. Наблюдатель состояния и моделирование}
Построим наблюдатель полного порядка для ОУ и исследуем замкнутую систему из задания~2b (следящий регулятор с внутренней моделью) при недоступности измерений переменных состояния объекта. Полюса наблюдателя размещены в нуле (deadbeat), матрица усиления $L$ рассчитана дуальным применением формулы Акерманна.

Уравнения наблюдателя:
\[
 \hat x_{k+1}=A_d \hat x_k + B_d u_k + L( y_k - C\hat x_k ),\qquad 
 e_k^{\text{obs}}=x_k-\hat x_k.
\]
Динамика ошибки наблюдения:
\[
e_{k+1}^{\text{obs}} = (A_d - LC) e_k^{\text{obs}}
\]
Выбор $L$ выполнялся из требования к характеристическому многочлену матрицы ошибок $A_d-LC$ (для deadbeat: $\lambda^n$). 

Дуальная формула Акерманна эквивалентна применению~\eqref{eq:ack} к паре $(A_d^T,C^T)$ и последующему транспонированию: 
\[
L=\bigl(e_n^T \mathcal O^{-1}\, \phi(A_d^T)\bigr)^T
\]
где $\mathcal O$~--- матрица наблюдаемости для пары $(A_d^T,C^T)$.

Для deadbeat-наблюдателя $\phi(\lambda)=\lambda^2$:
\[
\phi(A_d^T) = (A_d^T)^2 = \begin{bmatrix}1 & 2.4\\ 0 & 1\end{bmatrix}^2 = \begin{bmatrix}1 & 4.8\\ 0 & 1\end{bmatrix}
\]
\[
\mathcal O = \begin{bmatrix}C^T & A_d^T C^T\end{bmatrix} = \begin{bmatrix}0 & 0\\ 1 & 1\end{bmatrix}
\]
\[
L = \begin{bmatrix}0 & 1\end{bmatrix} \begin{bmatrix}0 & 0\\ 1 & 1\end{bmatrix}^{-1} \begin{bmatrix}1 & 4.8\\ 0 & 1\end{bmatrix} = \begin{bmatrix}0 & 1\end{bmatrix} \begin{bmatrix}0 & 0\\ 1 & 1\end{bmatrix} \begin{bmatrix}1 & 4.8\\ 0 & 1\end{bmatrix} = \begin{bmatrix}0 & 1\end{bmatrix}
\]

\subsection*{Эталонная модель и показатели качества (задание 3)}
В качестве эталонной модели для контура наблюдения задаётся 
\[\phi(\lambda)=\lambda^2\]
то есть желаемая динамика ошибки наблюдения описывается 
\[e^{\text{obs}}_{k+1} = F_e\, e^{\text{obs}}_{k},\quad F_e = A_d - LC,\quad \chi_{F_e}(\lambda) = \lambda^2,\]
что эквивалентно требованию \textbf{deadbeat} для наблюдателя: 
\[e^{\text{obs}}_{k+2} \equiv 0\quad \Rightarrow\quad t_s^{\text{obs}} = 2T = 1.50\,\text{с} \ (T=0.75\,\text{с}).\]

Показатели качества для контура наблюдения по результатам моделирования:
\begin{itemize}
    \item \textbf{Время установления (по ошибке наблюдения)}: $t_s^{\text{obs}} = 2T = 1.50$ с;
    \item \textbf{Перерегулирование}: 0\% (deadbeat);
    \item \textbf{Максимальная норма ошибки наблюдателя}: $\max_k \lVert e^{\text{obs}}_k \rVert \approx 2.60$; 
    \item \textbf{Установившееся значение ошибки наблюдателя}: 0.
\end{itemize}

Для контура слежения (регулятор задания из Задания~2 при замене $x$ на $\hat x$) эталонной является модель с \textbf{deadbeat}-размещением полюсов расширенной системы порядка 4, что задаёт 
время установления $t_s=4T=3.00$ с и нулевую установившуюся ошибку. Моделирование подтверждает достижение указанных показателей.

\textbf{Уравнения наблюдателя состояния:}
\[
\hat x_{k+1} = A_d \hat x_k + B_d u_k + L(y_k - C\hat x_k)
\]
\[
\hat x_{k+1} = \begin{bmatrix}1 & 0\\ 2.4 & 1\end{bmatrix} \hat x_k + \begin{bmatrix}0.75\\ 0.9\end{bmatrix} u_k + \begin{bmatrix}0\\ 1\end{bmatrix} (y_k - \begin{bmatrix}0 & 1\end{bmatrix} \hat x_k)
\]
где $\hat x_k = [\hat x_{1,k}, \hat x_{2,k}]^T$ — оценка состояния на $k$-м такте.

\textbf{Уравнение управления с наблюдателем:}
\[
u_k = -K_x \hat x_k - K_w w_k
\]
где $\hat x_k$ — оценка состояния объекта, $w_k$ — состояние внутренней модели (измеряется точно), $K_x=[2.4462\;\;1.3217]$, $K_w=[0.3943\;\;0.3399]$.

Результаты моделирования показывают:
\begin{itemize}
    \item Невязка наблюдателя $\lVert x-\hat x\rVert$ исчезает не более чем за два такта (deadbeat), на графике — практически за один;
    \item Выход $y(k)$ следует за гармоническим заданием $g(k)$ с нулевой установившейся ошибкой;
    \item Ошибка слежения $e(k)=g(k)-y(k)$ апериодически убывает и практически равна нулю к $t\approx 3$~с (четвёртый такт при $T=0.75$~с);
    \item На рисунке представлены $x_1, x_2$ и их оценки, выход и задание, состояние внутренней модели, невязка наблюдателя и ошибка слежения (масштаб по оси времени 0–10~с для ошибки).
\end{itemize}
Таким образом, после быстрого затухания невязки наблюдателя реализуется требуемое слежение с нулевой установившейся ошибкой.



\subsection*{Результаты моделирования (графики)}
На Рисунке~\ref{fig:task3_fast} показаны переменные состояния $x_1, x_2$ и их оценки, выход и задание, состояние внутренней модели, невязка наблюдателя, а также ошибка слежения с масштабом по времени 0–10~с. Видно, что невязка наблюдателя исчезает за $\leq 2$ такта, а ошибка слежения практически равна нулю к $t\approx 3$~с.

\begin{figure}[H]
    \centering
    \includegraphics{task3/fast_states.png}
    \caption{Быстрый регулятор: $x_1, x_2$ и их оценки, выход и задание, состояние внутренней модели, невязка наблюдателя и ошибка слежения}
    \label{fig:task3_fast}
\end{figure}

\section{Выводы}
Итоговые выводы:
\begin{itemize}
    \item (Задание 1) Стабилизирующий регулятор обеспечивает \textit{deadbeat}-сходимость объекта за 2 такта.
    \item (Задание 2) Следящая система с внутренней моделью обеспечивает нулевую установившуюся ошибку и \textit{deadbeat}-сходимость расширенной системы за 4 такта (3 секунды).
    \item (Задание 3) Наблюдатель полного порядка имеет эталонную \textit{deadbeat}-модель: невязка наблюдения зануляется не позднее чем за 2 такта (в моделировании — за 1 такт); ошибка слежения $e(k)=g(k)-y(k)$ апериодически затухает и практически равна нулю к $t\approx 3$~с.
\end{itemize}

Все расчёты выполнены аналитически с использованием формулы Акерманна. Для следящей части использовано $u_k = -\overline{K}\,\overline{x}_k$; для наблюдателя — дуальная постановка для выбора $L$.



